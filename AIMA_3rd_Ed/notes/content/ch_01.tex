% !TEX root = ../
\chapter{Introduction} % (fold)
\label{cha:introduction}

We've tried to understand how we think (perceive, understand, predict \&
manipulate). AI goes further and also tries to build these intelligent entities.

\section{What is AI?} % (fold)
\label{sec:what_is_ai_}
AI can be split into four sought after qualities in computers:
\begin{itemize}
    \item Thinking Humanly [Subsection~\ref{sub
:thinking_humanly}]
    \item Thinking Rationally [Subsection~\ref{sub
:thinking_rationally}]
    \item Acting Humanly [Subsection~\ref{sub
:acting_humanly}]
    \item Acting Rationally [Subsection~\ref{sub
:acting_rationally}]
\end{itemize}

\subsection{Acting Humanly (The Turing Test Approach)} % (fold)
\label{sub:acting_humanly}
The \uline{Turing Test} provides an operational definition of intelligence.
If a human interrogator cannot tell if the responses are from human or computer.
For now, a computer that passes needs the following capabilities:
\begin{enumerate}
    \item NLP - to communicate in English.
    \item Knowledge Representation - to store what it knows and hears.
    \item Automated Reasoning - reason on stored information to new conclusions.
    \item Machine learning to be adaptive and detect and extrapolate patterns.
\end{enumerate}
Turing avoids physical interaction because it is unrelated to intelligence.
A similar \uline{Total Turing Test} uses a video input as well.
Passing the Total Turing Test requires these extra skills:
\begin{enumerate}
    \setcounter{enumi}{4}
    \item Computer Vision - to perceive objects.
    \item Robotics - to manipulate objects.
\end{enumerate}
These six disciplines compose most of AI.

% subsection acting_humanly (end)

\subsection{Thinking Humanly (The Cognitive Modeling Approach)} % (fold)
\label{sub:thinking_humanly}

To write a program that thinks like a human, we need to understand human minds.
We can understand through introspection, psychological experiments, and brain
imaging.
Once we have a theory, we can implement it as a computer program.
Cognitive Science brings together Computer Models (AI) and experimental
psychology to understand how the human mind operates.
It is important to note that an algorithm behaving accurately doesn't imply it
uses a good model.

% subsection thinking_humanly (end)

\subsection{Thinking Rationally (The ``Laws of Thought'' Approach)} % (fold)
\label{sub:thinking_rationally}

Generally, rational thought is synonymous with logical thought.
We can create programs to compute logical correctness, but these programs have
bad problems (decidability being one of them).
It's hard to turn informal knowledge and state it in logical notation.
There is a difference between solving problems ``in principle'' and in practice.

% subsection thinking_rationally (end)

\subsection{Acting Rationally (The Rational Agent Approach)} % (fold)
\label{sub:acting_rationally}

\uline{Computer agents} are expected to autonomously perceive their environment,
persist over time, create and set goals.
\uline{Rational agents} act to achieve the best outcome.
In the ``laws of thought'' approach, emphasis is on inference, but that's only
part of being a rational agent.
Acting rationally doesn't always involve inference (i.e. reflexively recoiling
from a hot stove), and sometimes acting rationally occurs because something just
``must be done''. %TODO: rewrite this line?
Learning is not just for knowledge, but it also allows us to more effectively
behave rationally.
This approach has advantages over the other approaches.
It is more general than the ``laws of thought'' approach because it's more than
just correct inference.
Scientific development helps this approach more than the other developments.
Sometimes we can't always react 100\% rationally, as that is too computationally
expensive.
We assume that perfection is a good starting point for our analysis.
% subsection acting_rationally (end)

% section what_is_ai_ (end)

\section{Foundations Of Artificial Intelligence} % (fold)
\label{sec:foundations_of_artificial_intelligence}
This is a brief history of ideas that contribute to AI.
It glosses over many parts.

\subsection{Philosophy} % (fold)
\label{sub:philosophy}
Reasoning is like numerical computation, we can use formal rules to form
conclusions.

If the mind is governed by physical laws, then it has no more free will than a
rock ``deciding'' to fall toward the center of the earth.
Free will is the perception of choices appears to the choosing entity.
Nothing can be trusted from senses, because they can be lies.
General rules are acquired by exposure to repeated associations between their elements.
\uline{Logical positivism} states that all knowledge is logical theories of
connected observation sentences.

How does knowledge lead to action? \\
Aristotle defines that actions are justified by expected outcome and goals.
He defines an iterative algorithm that builds dependencies to achieve a goal.
% subsection philosophy (end)

\subsection{Mathematics} % (fold)
\label{sub:mathematics}
We need to develop logic, computation, and probability to implement philosophy.
Boolean logic is logic on propositions.
We can use first-order logic to include objects and relations.
These give us the formal rules to draw conclusions.

We have algorithms to prove statements.
There are true but undecidable statements.
Church-Turing thesis is Turing machines can compute all computable functions.
Intractable problems are ones in \mbox{EXPTIME}.
(Solvable only in exponential time with respect to input).

Probability allows us to reason with uncertain information.
Bayes rule underlies most modern approaches to reasoning in AI systems.
% subsection mathematics (end)

\subsection{Economics} % (fold)
\label{sub:economics}

Economies can be thought of as a individual agents maximizing their success.
People make choices that lead to preferred outcomes, or ``utility''.
Decision theory provides a framework for decisions made under uncertainty.
Some decisions have order associated with them - uses Marko Decision Processes.

% subsection economics (end)

\subsection{Neuroscience} % (fold)
\label{sub:neuroscience}
Neuroscience is the study of brains.
Localized areas of the brain are responsible for specific cognitive functions.
A collection of simple cells can lead to thought, action, and consciousness.
With computers of unlimited capacity, we still don't know how to make a brain.
% subsection neuroscience (end)

\subsection{Psychology} % (fold)
\label{sub:psychology}
Cognitive psychology views the brain as an information processing device.
Stimulus is translated into an internal model.
Internal models are manipulated by cognitive processes to form new models.
New models are translated into action.
% subsection psychology (end)

\subsection{Computer Engineering} % (fold)
\label{sub:computer_engineering}

Computer performance doubled every 18 months until 2005.
Power dissipation problems mean an increase cores instead of clock speed.
Work in AI owes much to CS and the converse is also true.
% subsection computer_engineering (end)

\subsection{Control Theory and Cybernetics} % (fold)
\label{sub:control_theory_and_cybernetics}

Controlling something is acting to minimize ``error'' (state vs goal).
Modern control theory designs systems to maximize a function over time.
Control theory is similar to AI because AI tries to maximize a function.
The two are different because controls are continuous, AI is otherwise.

% subsection control_theory_and_cybernetics (end)

\subsection{Linguistics} % (fold)
\label{sub:linguistics}

Understanding language requires an understanding of subject and context, not just sentence structure.
It's hard.

% subsection linguistics (end)

% section foundations_of_artificial_intelligence (end)

\section{The History of Artificial Intelligence} % (fold)
\label{sec:the_history_of_artificial_intelligence}

\subsection{The Gestation of Artificial Intelligence (1943-1955)} % (fold)
\label{sub:the_gestation_of_artificial_intelligence_}

Initially knowledge of physiology and propositional logic formed a basis for AI.
Artificial neurons were supposed to be ``on'' or ``off''.
Systems of neurons are Turing Complete [exciting!].
The \uline{Hebbian Rule} dictates connection strengths between neurons.
Alan Turing introduced the Turing Test.

% subsection the_gestation_of_artificial_intelligence_ (end)

\subsection{The Birth of Artificial Intelligence (1956)} % (fold)
\label{sub:the_birth_of_artificial_intelligence_}

Logic Theorist (program) was written to think non-numerically.
AI embraces creativity, self improvement, and language, it formed its own field
in cs.

% subsection the_birth_of_artificial_intelligence_ (end)

\subsection{Early Enthusiasm, Great expectations (1952 - 1969)} % (fold)
\label{sub:early_enthusiasm_great_expectations_}

Early AI research was successful because challenges were ``AI can't do X''.
GPS was written to be the first program to establish sub-goals - it
``thinks humanly''.
Using physical symbols as objects is necessary and sufficient for general AI.
Geometry Theorem Provider is made to prove tricky geometry problems.
A checkers agent was better than it's creator, disproving a theory.
Perceptron convergence theorem says that learning algorithms can learn for a
given input.

% subsection early_enthusiasm_great_expectations_ (end)

\subsection{A Dose of Reality (1966-1973)} % (fold)
\label{sub:a_dose_of_reality_}

Initially, AI researchers weren't shy about being optimistic.
Later on, they found they were running into three main problems:
First, the programs had no contextual basis for making decisions.
Second, AI solved by brute-forcing ``microworlds'', which is bad at scaling.
Genetic Algorithms were built because they seemed more minor, but they didn't
build good programs.
Third, the basic structures imposed limitations; simple perceptrons are limited.

% subsection a_dose_of_reality_ (end)

\subsection{Knowledge Based Systems: The Key to Power? (1969 - 1979)} % (fold)
\label{sub:knowledge_based_systems_the_key_to_power_}

Weak methods try to solve general problems by stringing together elementary
reasoning.
They fail for large problem spaces.
A better approach is to use domain-specific knowledge to help solve a problem.
Cook-book recipes helped solve hard problems from the basis of easy problems.
Heuristic programming was for investigating the capabilities of expert systems.
Expert systems are expensive, as they require a ton of information gathering.

% subsection knowledge_based_systems_the_key_to_power_ (end)

\subsection{AI Becomes an Industry (1980-Present)} % (fold)
\label{sub:ai_becomes_an_industry_}

In ~1989 nearly every US corp had an AI group and was using expert systems.
Later on, many companies failed on their promises.

% subsection ai_becomes_an_industry_ (end)

\subsection{The Return of Neural Networks (1986-Present)} % (fold)
\label{sub:the_return_of_neural_networks_}

Back propagation for neural nets was invented by at least four different groups.
Some think that humans manipulate symbols, others don't think connections, not
symbols are involved.
Current view is that they are complementary, but this question remains open.
Modern Neural Network is two fields:
One is for effective network architectures.
The other is concerned about empirical properties of actual neurons.

% subsection the_return_of_neural_networks_ (end)

\subsection{AI Adopts the Scientific Method (1987-Present)} % (fold)
\label{sub:ai_adopts_the_scientific_method_}

AI initially rebelled against Control Theory and Statistics, but is returning to
scientific rigor.
Speech recognition was tried with limited success in the past.
Hidden Markov Models now dominate speech recognition.
Hidden Markov Models are based on rigorous mathematical theory, and they can use
real-world data.
Bayesian networks allow fast representation of uncertain knowledge.

% subsection ai_adopts_the_scientific_method_ (end)

\subsection{The Emergence of Intelligent Agents (1995-present)} % (fold)
\label{sub:the_emergence_of_intelligent_agents}

Intelligent agents have been developed more recently to solve specific issues.
Some AI Giants (McCarthy, ++) believe that AI should go back to making
``machines that think''.
General Intelligence AI is ruled by a universal algorithm for learning and
acting.
General Intelligence AI is the golden goal.

% subsection the_emergence_of_intelligent_agents (end)

\subsection{The Availability Of Very Large Data Sets (2001-Present)} % (fold)
\label{sub:the_availability_of_very_large_data_sets_}

Some recent work says if we analyze more data, we don't need to focus on the
algorithm as much.
The idea is that you can learn context from large enough corpuses.
Small corpuses are pretty much expert systems.
Once AI passes the ``data bottleneck'', then it becomes scarily accurate.

% subsection the_availability_of_very_large_data_sets_ (end)

% section the_history_of_artificial_intelligence (end)

\section{The State of the Art} % (fold)
\label{sec:the_state_of_the_art}

AI can do tons today.
It can do driving, speech recognition, autonomous planning/scheduling, game
playing, spam fighting, logistics planning, robotics, machine translation, and
much more.
We will go into these in a much higher level of detail in future chapters.

% section the_state_of_the_art (end)

\section{Summary} % (fold)
\label{sec:summary}
Important points:
\begin{itemize}
    \item Intelligence is concerned with rational action.
    Agents try to make good choices.
    \item Economists formalized decision making.
    \item Neuroscientists discovered how the way brain works in contrast to
    computers.
    \item Computer Engineers provide powerful machines for AI.
    \item Control theory was the initial basis for AI.
    Initially, the two were very different, but they grow more similar.
    \item Recent progress on understanding intelligence has grown at pace with
    computer capabilities.
    \item The field of AI has grown substantially;
    sub-fields of AI are integral to other fields.
\end{itemize}
% section summary (end)

% chapter introduction (end)