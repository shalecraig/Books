% !TEX root = ../
\chapter{Intelligent Agents} % (fold)
\label{cha:intelligent_agents}

Chapter \ref{cha:introduction} referred to rational agents, here we define them.

\section{Agents and Environments} % (fold)
\label{sec:agents_and_environments}

Agents perceive their environment through sensors and act on it with actuators.
A ``percept'' is the agent's perceptual inputs at any given instant.
``Percept sequences'' are time-series of percept states.
``Agent functions'' describe the percept sequence to action mapping of an agent.
Agent functions are implemented as programs.
Agents are meant to be a tool for analyzing systems.

% section agents_and_environments (end)

\section{Good Behaviour - The Concept of Rationality} % (fold)
\label{sec:good_behaviour_the_concept_of_rationality}

Rational agents do the right thing, but what is the right thing?
Consequences of behaviour is a big measure.
Sometimes there is a performance measure to evaluate the environment state.
Some measures can be implemented dumbly, they need to be carefully designed.

\subsection{Rationality} % (fold)
\label{sub:rationality}
Rationality depends on four tenets:
\begin{enumerate}
    \item The performance that defines the criterion of success.
    \item The agent's prior knowledge of environment.
    \item The actions performable by the agent.
    \item The agent's percept sequence to date.
\end{enumerate}

Rational agents can be defined as follows:

\begin{em}
    For each possible percept sequence, a rational agent should select an action
    that is expected to maximize its performance measure, given the evidence
    provided by the percept sequence and whatever built-in knowledge the agent
    has.
\end{em}

It is necessary and sufficient that agents that fulfill all four tenets to be
rational.

% subsection rationality (end)

\subsection{Omniscience, Learning, and Autonomy} % (fold)
\label{sub:omniscience_learning_and_autonomy}

Omniscient agents are not possible within reality, because they \uline{know} the
outcome of their actions.
Rational agents draw conclusions about everything they \uline{can} know, not
everything knowable.
Rational agents then do information gathering to be better at predicting
consequences.
Rational agents should learn what they should expect to have to compensate for.
Applying learned knowledge is a prerequisite to success in many environments.

% subsection omniscience_learning_and_autonomy (end)

% section good_behaviour_the_concept_of_rationality (end)

\section{The Nature of Environments} % (fold)
\label{sec:the_nature_of_environments}

We know about rationality, let's talk about environment.

\subsection{Specifying the Task Environment} % (fold)
\label{sub:specifying_the_task_environment}

Task Environments are tuples of performance, environment, actuators, and
sensors. (PEAS)
Some task environments like the Roomba are very simple to define while others
like taxicab drivers are very hard.

% subsection specifying_the_task_environment (end)

\subsection{Properties of Task Environments} % (fold)
\label{sub:properties_of_task_environments}

We can categorize task environments into a few broad categories:
\begin{itemize}
    \item \uline{Fully v.s. Partially observable} environments - if the sensors
    can see everything, then the task is fully observable.
    \item \uline{Single Agent v.s. Multiagent} - if the agent is alone, or if
    there are multiple agents playing against it.
    \item \uline{Deterministic v.s. Stochaistic} - if the next step is
    completely determined by the current state and action, or if there is some
    sort of randomness involved.
    \item \uline{Episodic v.s. Sequential} -
    Episodic task environments are one-off episodes with consequences of actions.
    Sequential task environments are where short-term actions have consequences.
    \item \uline{Static v.s. Dynamic} -
    Static environments do not change while computation occurs.
    Dynamic environments  are where waiting for computation is choosing to
    ``do nothing''.
    \item \uline{Discrete v.s. Continuous} -
    State, time, and percepts can either be continuously occurring, or
    distinctly different states, times, and percepts.
    \item \uline{Known v.s. Unknown} -
    In known environments, outcomes are known.
    In unknown environments, agents will need to understand how it works to make
    good decisions.
\end{itemize}

Expectedly, the hardest environment is Partially Observable, Multiagent,
Stochaistic, Sequential, Dynamic, Continuous and Unknown.
Just like driving a car in an unfamiliar country, with unknown driving laws.

% subsection properties_of_task_environments (end)

% section the_nature_of_environments (end)

\section{The Structure of Agents} % (fold)
\label{sec:the_structure_of_agents}

% section the_structure_of_agents (end)

% chapter intelligent_agents (end)