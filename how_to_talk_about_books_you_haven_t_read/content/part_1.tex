Rough outline about why it's ok to lie about having read something, and how to do it.

There are two obligations:
1. {missed]
2. It's
3. We assume that you need to read a book to talk about it. This isn't true. Especially when someone hasn't read it either.

Reading is a risk, which is what we're going to talk about.

It's hard to find information about lieing about reading. Almost as bad as finding information about sex.

Sometimes we lie about reading books, so we don't have problems about this.

Makes an argument that ``reading a book'' is a fuzzy line. Did you read it closely? What is the representation of facts you read in it?

Many books that we haven't read have effect on us nonetheless, since the tentacles of their implications affect society.

1. Principle kinds of non-reading. There are levels of this. Books we've skimmed/heard about/forgotten are different things.
2. An exploration of situations where we talk about books we haven't read. (Apparently) this has examples.
3. Recommendations about how to lie about having read a book, from a lifetime of non-reading.

``The notion of the book that has been read'' is ambiguous.

There are four kinds of books:
- UB -> Unknown to Me
- SB -> Skimmed
- FB -> Forgotten
- ?? -> ??

The author will use these acronyms when citing sources, and will be explicit about having not read them. Interesting, eh?

It's rather grandiousie ``I hope one day these citation styles are widely used'' -- I wonder how successful this was.

The real value of books lies in their ability to conjure imaginations. By not reading it, you're able to use your imagination. I expect that this is going to be the crux of his argument, speckled with anecdotes about the way he once fooled someone into thinking that he'd read a book, or about the one time that he convinced someone to give him enough information about a book that he was abole to fabricate information about the book so that he had a cogent argument for (or against) the book's main theme, in a way that transcended the arguments of the person that he was talking about.

This is going to be interesting... I wonder if I'm going to continue taking notes while listening to audio books..

On the surface, taking notes while in this site feels like a good idea. Transcription of thoughts, and stuff, but the flipside is that I'm forced to do this.

It wasn't too bad, was it?
